\subsection*{Causal and Normative Concepts}

\begin{description}[style=nextline,leftmargin=1.5cm]

      \item[Causal Commitment]
            An ontological commitment that asserts causal relations
            (e.g., A caused B) as foundational facts.
            Causal commitments are model-dependent
            and vary across admissible causal frameworks.

      \item[Deontic Commitment]
            An ontological commitment to obligations, permissions, prohibitions,
            or duties as foundational facts.
            Deontic commitments specify what ought to be done,
            not merely what exists or occurs.

      \item[Framework-Dependent]
            A property of claims or predicates whose truth value
            varies across admissible interpretive frameworks.
            Causal and normative claims are framework-dependent;
            entity existence and identity conditions are not.

      \item[Normative Commitment]
            An ontological commitment that encodes evaluative, legal, moral,
            or policy-based judgments as part of the foundational representation.
            Normative commitments may be ethical, legal, institutional, or procedural,
            and are actor- or framework-relative.

      \item[Pre-Causal]
            The property of an ontology that excludes causal primitives
            from its foundational layer.
            Pre-causal does not mean acausal or causation-denying;
            it means \emph{prior to causal attribution}.
            A pre-causal substrate represents entities and events
            without asserting causal relations between them,
            allowing causal reasoning to occur in interpretive layers
            where model assumptions are explicit and disagreement is representable.
            Causation is externalized, not eliminated.

      \item[Pre-Normative]
            The property of an ontology that excludes normative primitives
            from its foundational layer.
            A pre-normative ontology represents entities and actions
            without asserting obligations, permissions, or evaluations.

\end{description}
