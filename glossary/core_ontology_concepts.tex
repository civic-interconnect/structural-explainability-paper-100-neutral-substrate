\subsection*{Core Ontology Concepts}

\begin{description}[style=nextline,leftmargin=1.5cm]

    \item[Applied Ontology]
          The use of ontological methods to structure and analyze real-world domains
          for practical purposes.

    \item[Boundary Condition]
          A precondition that must hold for a framework or result to apply.
          In this paper, the central boundary condition is that
          entity existence and identity conditions must be invariant
          across admissible frameworks within the domain.
          Domains where identity is itself contested
          do not admit neutral substrates in the sense defined here.
          Boundary conditions delimit scope;
          they are not defects but explicit constraints on applicability.

    \item[Embedding]
          The act of asserting a claim as a foundational fact within an ontology,
          such that it becomes part of the substrate-layer representation.
          Embedded content is treated as true by the ontology itself,
          not merely recorded as a claim made by some agent.
          Contrast with \emph{Reification}.

    \item[Epistemic Ontology]
          Ontologies that represent knowledge about the world
          rather than the world directly.
          Epistemic ontologies encode claims, beliefs, or assertions
          as reified entities,
          allowing representation of diverse perspectives
          without embedding contested conclusions
          as substrate-layer commitments.
          Contrast with \emph{Metaphysical Ontology}.

    \item[Existence Conditions]
          The criteria that determine whether an entity is present in a domain.
          Existence conditions specify what must hold for an entity
          to be recognized by the ontology,
          independent of how it is interpreted or evaluated.

    \item[Identity Conditions]
          The criteria that determine when two references
          denote the same entity, and what changes an entity
          can undergo while remaining the same entity.
          Identity conditions establish persistence and individuation
          prior to interpretive or causal analysis.

    \item[Level of Abstraction (LoA)]
          A perspective defined by a specific set of observables
          used to represent and analyze a domain.
          The LoA determines what can be seen or queried;
          selecting a restricted LoA excludes
          certain predicates from the representation.

    \item[Level of Organization (LoO)]
          The architectural layer or scale at which entities and relationships
          are modeled within a system.
          Levels of Organization are structural rather than scalar:
          they differ by direction of dependence (e.g., foundational versus interpretive),
          not by degree, granularity, or complexity.
          A system may have multiple LoOs arranged hierarchically,
          with foundational layers supporting higher-order interpretive layers.

    \item[Metaphysical Ontology]
          An ontology that represents the world directly,
          asserting what exists and what relations hold
          as ontological facts.
          Metaphysical ontologies embed commitments
          about the structure of reality at the substrate layer.
          Contrast with \emph{Epistemic Ontology}.

    \item[Observables]
          The set of properties, relations, or predicates
          that an ontology makes available for representation and query.
          An ontology's Level of Abstraction is defined by its observables;
          neutrality requires restricting observables to those
          invariant across admissible frameworks.

    \item[Ontological Assertion]
          A statement embedded in an ontology as a foundational fact.
          Ontological assertions carry ontological commitment:
          they represent the ontology's position on what exists or holds,
          not merely what has been claimed by external agents.

    \item[Ontology]
          A formal representation of entities within a domain
          and the relationships that hold between them,
          expressed in a logic or schema suitable for inference.

    \item[Reification]
          The representation of a claim, assertion, or relation
          as an entity rather than as a direct ontological fact.
          Reification allows the ontology to represent discourse
          about the world without asserting conclusions about the world itself,
          for example, by representing that an agent asserted
          a causal or normative claim without embedding that claim
          as a substrate-layer fact.
          See also \emph{Epistemic Ontology}.
          Contrast with \emph{Embedding}.

    \item[Representation]
          The encoding of information within an ontology.
          Representation is broader than embedding:
          an ontology may represent a claim (via reification)
          without embedding it as a foundational fact.
          See also \emph{Embedding}, \emph{Reification}.

    \item[Substrate]
          The foundational layer of an ontology,
          consisting of entities, identity conditions, and persistence criteria.
          A neutral substrate excludes causal and normative primitives,
          serving as a stable base for divergent interpretive layers.
          Also called \emph{substrate layer} or \emph{foundational layer}.

\end{description}

