\subsection*{Neutrality and Interpretation}

\begin{description}[style=nextline,leftmargin=1.5cm]

    \item[Accountability System]
          A data system designed to support explanation, evaluation,
          and assignment of responsibility across multiple stakeholders.
          Accountability systems must accommodate persistent disagreement
          about causes, obligations, and interpretations.

    \item[Admissible Framework]
          An interpretive framework that is internally consistent
          and represents a legitimate stance within the domain of application,
          such as a recognized legal jurisdiction,
          an established causal modeling tradition,
          or a coherent normative theory.
          Admissible frameworks need not be mutually compatible.

    \item[Commitment]
          An ontological stance that asserts
          certain facts, relations, or classifications
          as true at the substrate layer.
          Commitments constrain what interpretive frameworks
          can be layered atop the ontology without contradiction.

    \item[Compatibility]
          The property of an ontology and an interpretive framework
          such that their combination does not entail contradiction.
          Formally: $\mathcal{S} \cup \mathcal{F} \not\vdash \bot$.

    \item[Consistency ($\not\vdash \bot$)]
          The property of a set of statements such that no contradiction
          can be derived from its assertions.
          The notation $\mathcal{S} \not\vdash \bot$ indicates
          that an ontology $\mathcal{S}$ is consistent;
          $\mathcal{S} \vdash \bot$ indicates that $\mathcal{S}$ is inconsistent
          (i.e., both $p$ and $\neg p$ can be derived for some proposition $p$).

    \item[Extension Stability]
          The requirement that a substrate remain logically consistent
          when extended by admissible, including mutually incompatible,
          interpretive frameworks.
          Violation occurs when adding a framework forces revision
          of substrate-layer assertions.
          Together with \emph{Interpretive Non-Commitment}, defines \emph{Neutrality}.

    \item[Interpretive Framework]
          A coherent set of principles, rules, or assumptions
          used to evaluate, explain, or assign meaning to entities and events.
          Examples include legal jurisdictions, causal models,
          ethical theories, and analytic methodologies.
          Interpretive frameworks may disagree with one another
          while each remaining internally consistent.

    \item[Interpretive Non-Commitment]
          The requirement that a substrate refrain from asserting conclusions
          whose truth varies across admissible interpretive frameworks.
          An ontology satisfies interpretive non-commitment
          if it does not privilege one framework's conclusions over another's.
          Together with \emph{Extension Stability}, defines \emph{Neutrality}.

    \item[Neutral Substrate]
          A foundational ontology that satisfies neutrality,
          enabling stable use across multiple,
          potentially incompatible interpretive frameworks.
          A neutral substrate excludes causal and normative commitments
          at the foundational layer,
          supporting pluralism and accountability.

    \item[Neutrality]
          The property of an ontology that avoids embedding causal, deontic,
          or normative commitments at the foundational layer,
          enabling stable use across persistent interpretive, legal,
          and analytic disagreement.
          Neutrality is defined by two requirements:
          interpretive non-commitment and extension stability.

    \item[Pluralism]
          The coexistence of multiple, potentially incompatible
          interpretive frameworks within a single system.
          A neutral substrate supports pluralism
          by providing a shared foundation that does not
          privilege any single framework.

    \item[Pluralism Collapse]
          The loss of pluralism that occurs when an ontology
          embeds commitments that exclude one or more admissible frameworks.
          Pluralism collapse reduces the system to a single privileged interpretation.

\end{description}

