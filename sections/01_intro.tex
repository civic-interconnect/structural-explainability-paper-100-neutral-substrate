% !TeX root = se100_the_ontological_neutrality_theorem.tex

\section{Introduction: Accountability, Disagreement, and Neutral Substrates}
\label{sec:introduction}

Data systems operate in environments characterized
by persistent disagreement~\cite{rawls1993political,sunstein1996legal}.
Legal interpretations diverge across jurisdictions and
over time~\cite{dworkin1986law,hart1961concept,berman2007pluralism};
political actors contest responsibility
and intent~\cite{bovens2007accountability,hood2011blame};
analytic frameworks disagree about causation, attribution,
and relevance~\cite{pearl2009causality,woodward2003making}.
These disagreements are not anomalies to be resolved prior to data representation.
They are structural features of modern industry and society
that any accountability-oriented system must accommodate.

Accountability, in this context,
requires more than the collection of records or
the publication of datasets~\cite{fox2007transparency,bovens2014oxford}.
It requires the ability to refer stably to
entities, actions, and institutional artifacts
while allowing competing explanations, evaluations,
and judgments to coexist~\cite{gruber1993translation,smith2010ontology}.
A system that collapses disagreement by embedding a single interpretation
into its foundational representations undermines
accountability rather than supporting it,
as it forecloses contestation and
revision~\cite{bowker1999sorting,winner1980artifacts}.

This paper is concerned with the implications
of this requirement for ontology design.
Specifically, it addresses the conditions under which an ontology
can function as a neutral substrate for accountability:
a shared representational base that remains usable across
incompatible legal, political, and analytic frameworks.
Under this conception of neutrality,
such a substrate must support the coexistence of
divergent interpretations
without requiring revision of the substrate itself.
This requires a careful selection of the
Level of Abstraction (LoA)~\cite{floridi2008levels}
at which the substrate operates.
Following Floridi, we argue that an LoA is
defined by the set of observables it makes available;
for a substrate to remain neutral,
it must operate at an LoA that excludes
explanatory or evaluative observables.

The central claim of this paper is that this form of neutrality
imposes strict design constraints.
In particular, ontologies that assert causal or normative conclusions
at the foundational layer cannot satisfy the requirements
of interpretive non-commitment and extension stability that neutrality
demands~\cite{hacking1999social,cartwright1979causal,star1996ecology,guarino2009ontology}.
When causal relations or deontic judgments are embedded as ontological facts,
the substrate necessarily privileges one framework over others,
rendering it incompatible with at least some admissible interpretations.

Rather than proposing an ontology or protocol,
this paper establishes these constraints at a meta-level.
It formalizes neutrality in terms of interpretive non-commitment and stability
under incompatible extensions, and shows that these requirements
are incompatible with causal or normative commitments at the substrate layer.
The result is an impossibility claim:
no ontology that embeds contested causal or deontic conclusions
can serve as a neutral substrate for accountability across disagreement.

The remainder of the paper proceeds as follows.
Section~\ref{sec:related} reviews related work in ontology design,
legal informatics, and causal modeling, situating the present contribution.
Section~\ref{sec:requirements} defines the notion of a neutral substrate and
specifies the two neutrality requirements relevant to accountability systems.
Section~\ref{sec:impossibility} presents an impossibility argument
showing why causal and normative commitments violate these requirements.
Section~\ref{sec:implications} discusses the implications of this result for ontology design,
clarifying what must be excluded from foundational layers
and what may be externalized to interpretive frameworks.
Section~\ref{sec:conclusion} summarizes the resulting constraints
and discusses their relevance for ontology design in accountability contexts.
