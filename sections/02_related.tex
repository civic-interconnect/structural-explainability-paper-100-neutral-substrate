% !TeX root = se100_the_ontological_neutrality_theorem.tex

\section{Related Work}
\label{sec:related}

The constraints developed in this paper intersect with several lines of prior work:
upper ontologies in formal ontology research,
domain ontologies for legal and institutional reasoning,
and causal modeling frameworks.
This section situates the present contribution
relative to each, clarifying both debts and departures.

\subsection{Political Philosophy: Reasonable Pluralism}

The premise that disagreement is structural rather than resolvable
draws on Rawls' concept of reasonable pluralism~\cite{rawls1993political}.
Rawls argues that under conditions of free inquiry,
reasonable persons will arrive at incompatible comprehensive doctrines.

This paper transposes that insight to ontology design:
if interpretive disagreement is a permanent feature of accountability contexts,
then neutral substrates must accommodate it structurally
rather than resolving it by fiat.
The impossibility result formalizes this accommodation
as a constraint on admissible observables.
Rawls' argument is not imported as a normative foundation,
but as an analogy for the persistence of disagreement
that motivates the need for structural accommodation.

\subsection{Upper Ontologies: BFO and DOLCE}

The Basic Formal Ontology
(BFO)~\cite{iso2021bfo,arp2015bfo} and
the Descriptive Ontology for Linguistic and Cognitive Engineering
(DOLCE)~\cite{borgo2022dolce,masolo2003dolce}
represent two influential approaches to foundational ontology design.
Both aim to provide domain-independent upper-level categories
that can be extended for specific applications.
A recent special issue~\cite{borgo2022foundational} provides systematic
comparison of seven foundational ontologies
(BFO, DOLCE, GFO, GUM, TUpper, UFO, YAMATO)
through common modeling cases,
illustrating both their shared commitments
and divergent design choices.

BFO distinguishes continuants from occurrents
and provides a realist framework grounded in scientific practice.
DOLCE takes a more cognitive and linguistic orientation,
emphasizing the role of conceptualization in ontological commitment.
Both have been widely adopted in biomedical, engineering,
and information systems contexts.

The present work addresses a different question:
it analyzes the logical consequences
of embedding interpretive commitments at the substrate level,
rather than the adequacy or intent
of any particular foundational ontology.
BFO and DOLCE provide categorical structure for what exists;
they do not explicitly address the conditions under which
an ontology can remain neutral across interpretive disagreement.
These frameworks are often extended in practice
to include causal and normative relations
as first-class ontological commitments.
The impossibility result established here
implies that such extensions, if embedded at the substrate layer,
compromise neutrality in the sense required for accountability systems.

In Floridi's terms, while BFO and DOLCE define
the Level of Organization for physical and social reality,
they do not mandate a specific Level of Abstraction
that excludes interpretive predicates.
Our contribution is to show that for the
specific purpose of neutrality,
such a mandate is a logical necessity.

Guarino's analysis of ontological commitment~\cite{guarino2009ontology}
clarifies what it means for an ontology to assert something;
the present contribution identifies which such assertions
are incompatible with neutrality.

This paper's constraints are therefore compatible
with BFO or DOLCE
as upper-level scaffolding,
provided that causal and normative commitments are externalized
to interpretive layers rather than embedded in the foundational substrate.
Recent work on formal alignment between BFO and DOLCE~\cite{masolo2025alignment}
demonstrates that even foundational ontologies with divergent commitments
can be mapped systematically,
reinforcing the feasibility of neutral substrates
that support multiple interpretive frameworks.

Whether specific versions or extensions of BFO, DOLCE,
or other upper ontologies satisfy these constraints in practice
is an empirical question beyond the scope of this paper.
Such compatibility assessments require detailed analysis
of particular ontology versions and their commitments,
work that would benefit from collaboration with the communities
that maintain and extend these frameworks.


\subsection{Legal Ontologies: LKIF and Related Approaches}

Legal ontologies, such as the Legal Knowledge Interchange Format
(LKIF)~\cite{hoekstra2007lkif},
and related frameworks~\cite{sartor2011legal,breuker2009owl}
aim to represent legal concepts, norms, and reasoning patterns
in machine-processable form.
These ontologies typically include deontic categories, such as obligation,
permission, and prohibition, as core primitives,
along with relations for legal causation, responsibility, and compliance.

Such frameworks are valuable for legal reasoning and case analysis
within a fixed normative framework.
However, they are not designed to function as neutral substrates
across incompatible legal or political interpretations.
By embedding deontic conclusions as ontological facts at the foundational layer,
they necessarily privilege one normative stance over others.
When legal interpretations diverge,
as they routinely do across jurisdictions, over time,
or under political contestation,
the substrate itself must be revised
to accommodate disagreement.

The impossibility result in this paper
applies directly to such designs.
LKIF and similar ontologies are appropriate for
interpretive layers built atop a neutral substrate,
but they cannot themselves serve as the substrate
if neutrality across disagreement is required.

\subsection{Causal Modeling: Pearl and Structural Approaches}

The structural causal modeling framework developed by Pearl~\cite{pearl2009causality}
and extended by others~\cite{woodward2003making,spirtes2000causation}
provides rigorous tools for representing and reasoning about causation.
The do-calculus and related interventionist semantics
have become standard in causal inference across multiple disciplines.

These frameworks make explicit causal commitments:
they assert that certain variables cause others
under specified structural assumptions.
Such commitments are essential for causal reasoning
but are inherently model-dependent.
Different causal models, each internally consistent,
may disagree about which relations are causal,
which variables are confounders,
and which interventions produce which effects.

The present work does not dispute the value of causal modeling
for explanation and decision-making.
Rather, it observes that causal conclusions are framework-relative
in the same structural sense as normative conclusions.
Embedding causal relations as substrate-layer primitives
privileges one model over others,
violating interpretive non-commitment.
Causal reasoning must therefore be externalized
to interpretive layers where model assumptions are explicit
and disagreement can be represented without substrate revision.

\subsection{Philosophy of Information: Levels of Abstraction}

The methodological framework of this paper is informed by Floridi's
Philosophy of Information~\cite{floridi2008levels,floridi2011philosophy}.
Floridi introduces the Method of Levels of Abstraction (LoA) as a way to
specify the epistemic range of a system through a set of observables.
Crucially, Floridi distinguishes between
the LoA (the lens through which a system is viewed) and
the Level of Organization (LoO), which refers to
the internal structural hierarchy of the system itself.

While Floridi uses these concepts
to analyze the nature of data and knowledge, this paper applies
them specifically to the problem of ontological neutrality.
We argue that neutrality is a function of
operating at a foundational LoO
with a deliberately restricted LoA.
This paper extends Floridi's work by proving
that certain classes of observables, causal and normative,
necessarily introduce interpretive entropy that
renders an LoA unstable with respect to extensions.

\subsection{Information Infrastructure: Boundary Objects}

Star and Griesemer's concept of
boundary objects~\cite{star1989institutional,star1996ecology}
and subsequent work on classification systems~\cite{bowker1999sorting}
address how representations can be shared across communities
with divergent interpretive frameworks.
Boundary objects are plastic enough to adapt to local needs
yet robust enough to maintain identity across sites.

The neutral substrate concept developed here
can be understood as a formalization of this insight:
the substrate provides the shared identity structure,
while interpretive frameworks supply local meaning.
The impossibility result specifies what cannot be shared
at the boundary level without losing plasticity.

\subsection{Positioning the Present Contribution}

The contribution of this paper is a meta-theoretical constraint.
Unlike BFO and DOLCE, it does not propose a specific taxonomy of entities;
rather, it defines the informational boundary
that any such taxonomy must respect to remain neutral.

Recent work has questioned the stability of top-level ontological
commitments in emerging technical domains.
Köhler and Neuhaus argue that large language models
exhibit a \textit{mercurial} ontological status:
their classification shifts depending on task framing,
deployment context,
and explanatory purpose~\cite{kohler2025mercurial}.
This observation supports the present result by illustrating
that even top-level ontological categories
cannot be guaranteed to remain stable
when interpretive predicates are embedded,
reinforcing the need, where neutrality across uses is required,
for a neutrality-preserving substrate
beneath such shifting conceptualizations.

Recent work also emphasizes that ontology development itself
requires negotiating consensus among stakeholders
with divergent commitments~\cite{neuhaus2022consensus}.
This observation reinforces the present argument by contrast:
while temporary disagreement during ontology construction
may be instrumental in converging on a sound design,
ontologies intended to operate after construction
in contexts of valid and persistent interpretive disagreement
among admissible frameworks
cannot rely on consensus as a foundational design assumption.
Instead, such ontologies must, in these contexts, be structured to
accommodate disagreement structurally without requiring its resolution.
The impossibility result specifies
the formal constraint such accommodation must satisfy.

Recent ontology engineering work has emphasized
tool-supported maintenance and rapid evolution in fast-moving domains.
For example, the Artificial Intelligence Ontology (AIO) employs
LLM-assisted curation and automated ODK pipelines
to keep a domain ontology current
as AI concepts and ethical concerns evolve~\cite{joachimiak2025aio}.
Such approaches highlight the importance of distinguishing
between evolving domain-level ontologies
and the comparatively stable substrates on which they depend:
the present result explains why neutrality constraints must apply
to the latter even as the former necessarily change.

By grounding this constraint in the Method of Levels of Abstraction,
we move the debate from a sociological discussion of agreement
to a formal requirement for observational invariance.

This result is complementary to prior work rather than competitive.
Upper ontologies may inform the categorical structure of a neutral substrate.
Legal and causal frameworks may operate as interpretive layers atop it.
What the present work provides is a principled boundary:
a specification of what must be externalized
in order for the substrate to support accountability across disagreement.
