% !TeX root = se100_the_ontological_neutrality_theorem.tex

\section{Formal Requirements for Ontological Neutrality}
\label{sec:requirements}

The notion of neutrality used in this paper
is deliberately narrow and operational.
It is not intended to capture moral neutrality, political impartiality,
or epistemic skepticism in a general sense.
Rather, neutrality is defined with respect to the functional role an ontology
is expected to play in explanation and accountability systems:
serving as a shared substrate across divergent interpretive frameworks.

An ontological substrate is neutral, in the sense relevant here,
if it satisfies two requirements.
These requirements are independent but jointly necessary.
Together, they characterize the minimal conditions
under which a shared ontology can support accountability
without foreclosing disagreement.

Throughout this section, all requirements are understood
as analytic consequences of this
role-relative definition of neutrality,
not as general design prescriptions for ontologies.

Interpretive non-commitment constrains what the substrate asserts
at its chosen Level of Abstraction.
Extension stability is a global consistency condition:
even a substrate that is silent on contested predicates
can fail neutrality if it encodes background axioms
that conflict with some admissible interpretive framework.


%-------------------------------------------------------
\subsection{Interpretive Non-Commitment}
\label{subsec:interpretive-nc}
%-------------------------------------------------------
\REQ{INC}{The substrate MUST exclude causal and normative observables
  from its Level of Abstraction.}
\WHY{INC}{Causal and deontic observables vary across admissible interpretive frameworks
  and therefore violate neutrality.}

The first requirement,
under this definition of neutrality,
is interpretive non-commitment.
An ontology satisfies this requirement
if it does not assert conclusions
whose truth values vary across admissible interpretive frameworks.

Admissibility is determined by the domain of application,
not by the substrate.
A framework is admissible if recognized stakeholders
in the domain accept it as a legitimate interpretive stance,
for example, a jurisdiction's legal code,
a scientific community's causal methodology,
or an institution's normative charter.
The substrate must accommodate all such frameworks;
it does not adjudicate among them.

\begin{definition}[Admissible Framework]
  A framework $\mathcal{F}$ is \emph{admissible} if it is internally consistent
  ($\mathcal{F} \not\vdash \bot$)
  and represents a legitimate interpretive stance within the domain of application.
  The set of all admissible frameworks is denoted $\mathbb{F}$.
  Admissible frameworks need not be mutually compatible.
\end{definition}

\noindent
With this notion in hand, we can state the first requirement precisely.

\begin{definition}[Framework-Variant Proposition]
  A proposition $p$ is \emph{framework-variant}
  if there exist admissible frameworks
  $\mathcal{F}_1, \mathcal{F}_2 \in \mathbb{F}$
  such that
  $\mathcal{S} \cup \mathcal{F}_1 \vdash p$
  and
  $\mathcal{S} \cup \mathcal{F}_2 \vdash \neg p$.
\end{definition}


\begin{definition}[Interpretive Non-Commitment]
  An ontology $\mathcal{S}$ satisfies \emph{interpretive non-commitment}
  if there is no proposition $p$ such that $\mathcal{S} \vdash p$
  and the truth of $p$ varies across admissible interpretive frameworks.
\end{definition}

Accountability systems routinely involve disagreements
about what occurred, why it occurred, and how it should be evaluated.
Legal frameworks may disagree about whether an action was permitted or prohibited;
analytic frameworks may disagree about whether one event caused another;
political frameworks may disagree about responsibility or intent.
These disagreements are not merely epistemic gaps
to be filled with additional data.
They reflect differences in background assumptions,
governing rules, and interpretive commitments.

A neutral substrate,
under this definition,
must therefore refrain from
settling such questions at the ontological level.
If an ontology asserts that a particular action was
obligatory, forbidden, or permitted,
it necessarily privileges one normative framework over others.
Similarly, if it asserts that one event caused another,
it privileges a particular causal model or explanatory framework.
In both cases, the ontology ceases to be neutral with respect to interpretation.

Interpretive non-commitment does not require the ontology
to ignore normative or causal discourse altogether.
Rather, it requires that the substrate operate
at a Level of Abstraction (LoA)~\cite{floridi2008levels,floridi2011philosophy}
that excludes explanatory or evaluative observables.
In this framework, an LoA consists of a set of observables;
neutrality is achieved by selecting an LoA
where the observables are restricted to
entity existence and identity conditions
(i.e., what entities exist and how they are individuated).

These observables constitute the
invariant core of the representation,
remaining valid regardless of the
higher-level interpretive logic applied to them.
Claims about obligations, permissions, or causation
are thus represented as external assertions
made by agents within an interpretive framework,
rather than being internalized as substrate-layer facts.

\OBS{INC-DISCHARGE}{This subsection discharges the
  interpretive non-commitment component of neutrality
  by restricting admissible observables.}

%-------------------------------------------------------
\subsection{Extension Stability}
\label{subsec:extension}
%-------------------------------------------------------
\REQ{EXT}{The substrate MUST remain stable under extension
  by incompatible interpretive frameworks.}

\WHY{EXT}{Neutral substrates MUST support disagreement
  without revision of foundational commitments.}


The second requirement,
given the neutrality role defined above,
is extension stability.
An ontology satisfies this requirement if it remains consistent
when extended by incompatible interpretive frameworks.

\begin{definition}[Extension Stability]
  An ontology $\mathcal{S}$ satisfies \emph{extension stability}
  if for all admissible frameworks $\mathcal{F} \in \mathbb{F}$,
  $\mathcal{S} \cup \mathcal{F} \not\vdash \bot$.
\end{definition}

Extension stability does not require that all admissible interpretive frameworks
be simultaneously combined into a single, globally consistent theory.
Rather, the requirement is \emph{pairwise compatibility with the substrate}:
for each admissible framework $\mathcal{F} \in \mathbb{F}$ considered independently,
the combined theory $\mathcal{S} \cup \mathcal{F}$ must remain consistent.
The substrate is permitted to be extended by additional structures,
records, or interpretive layers, provided such extensions do not retract,
revise, or contradict substrate-layer assertions.
A violation occurs only when accommodating a new admissible framework
requires modification of the substrate itself.
Disagreement between frameworks, including disagreement about causal,
normative, or explanatory conclusions, is expected and permitted;
what is prohibited is embedding commitments at the substrate layer
that force revision when such disagreement arises.

\OBS{EXT-SCOPE}{Extension stability is a constraint on the substrate,
  not a requirement that interpretive frameworks be mutually compatible.}

In practice, data systems are not static.
New legal interpretations emerge, analytic models are revised,
and political judgments are contested and overturned.
A neutral substrate,
in order to satisfy extension stability,
must be able to accommodate such changes
without requiring revision of its foundational commitments.
Interpretive disagreement should result in additional structure
layered atop the substrate, not in modification of the substrate itself.

Extension stability is violated when
the introduction of a new interpretive framework
requires retracting or revising ontological assertions.

Consider a substrate $\mathcal{S}$ used by two interpretive frameworks,
$\mathcal{F}_1$ and $\mathcal{F}_2$, that contradict each other
($\mathcal{F}_1 \cup \mathcal{F}_2 \vdash \bot$).
For $\mathcal{S}$ to remain stable, it must be separately consistent
with each framework:
$\mathcal{S} \cup \mathcal{F}_1 \not\vdash \bot$ and
$\mathcal{S} \cup \mathcal{F}_2 \not\vdash \bot$.
The substrate need not reconcile the frameworks with each other;
it must simply avoid assertions that either framework rejects.

Now suppose $\mathcal{S}$ embeds a causal or normative conclusion $c$
that $\mathcal{F}_1$ accepts but $\mathcal{F}_2$ rejects
(i.e., $\mathcal{F}_2 \vdash \neg c$).
Then $\mathcal{S} \cup \mathcal{F}_2 \vdash c \land \neg c \vdash \bot$:
the combination is inconsistent, and $\mathcal{S}$ must be revised.

This demonstrates that embedding any causal or deontic conclusion,
regardless of current consensus,
introduces latent instability.
Such a substrate cannot function as a permanent neutral foundation,
because some admissible framework may eventually reject
what the substrate asserts.

This requirement rules out approaches in which neutrality
is achieved through periodic revision or replacement of ontological commitments.
Stability under extension is a necessary condition
for interoperability across time, jurisdictions, and analytic communities.

This property is sometimes called \emph{monotonicity}:
the substrate grows by addition of interpretive layers,
never by retraction of foundational assertions.

\OBS{EXT-DISCHARGE}{This subsection formalizes instability
  introduced by causal or normative primitives
  under incompatible extensions.}

%-------------------------------------------------------
\subsection{Scope of Neutrality}
\label{subsec:scope}
%-------------------------------------------------------

\REQ{SCOPE-GUARD}{This paper MUST NOT argue that all ontologies should be neutral,
  only that neutral substrates have categorical constraints.}
\WHY{SCOPE-GUARD}{Many ontologies are legitimately causal or normative;
  neutrality is a role-relative requirement, not a universal virtue.}


It is important to emphasize that neutrality,
as defined here, is not global.
A neutral ontology is not neutral with respect to
existence, identity, or classification.
It must take substantive positions on what kinds of entities exist,
how they are distinguished, and what criteria govern their persistence.
These commitments are necessary for accountability
and do not vary across interpretive frameworks
in the same way that causal or normative conclusions do.
Unless otherwise noted, all entailment is with respect to
classical first-order logic with equality.

Neutrality is therefore scoped specifically to interpretive commitments:
those aspects of representation whose truth depends on
legal, normative, causal, or analytic frameworks
that may legitimately disagree.
The exclusion of such commitments from the substrate
is not a matter of preference or conservatism;
it is a functional requirement derived from
the role the ontology is intended to play
as a neutral substrate.

The design of a neutral substrate requires
distinguishing between the Level of Abstraction (LoA) and
the Level of Organization (LoO) as referenced above.
While the LoA determines the set of available observables
(e.g., excluding causal predicates),
the LoO characterizes the architectural hierarchy of the system.

A further scope condition applies.
The neutrality requirements assume that entity existence and identity conditions
are themselves invariant across admissible frameworks.
If frameworks disagree about identity, for example,
whether two references denote the same actor,
or whether a particular institutional artifact exists, then
the substrate cannot be neutral with respect to those frameworks.
In such cases, identity must be resolved at a prior stage,
or the domain does not admit a neutral substrate in the sense defined here.
This is not a defect of the framework but a boundary condition:
neutrality presupposes a shared ontological ground on which interpretation can vary.

\OBS{ID-INVAR-DISCHARGE}{This subsection establishes that
  identity invariance is a boundary condition for neutrality,
  discharging REQ{ID-INVAR}.}

We propose that the neutral substrate occupies the primary LoO,
providing the raw entity and identity conditions
upon which higher-level, interpretive LoOs are organized.
This ordering reflects the direction of dependence
required for neutrality:
identity is established at the substrate layer
before any interpretive framework is applied.
Explanation and interpretation operate on entities
whose existence and identity are already fixed.
This ensures that changes in higher-level organizational logic
do not propagate downward to corrupt the foundational substrate.

\OBS{ID-DISCHARGE}{This subsection establishes that
  entity existence and identity conditions are
  fixed at the substrate layer,
  prior to interpretive extension,
  discharging REQ{ID}.}

\OBS{DEP-DISCHARGE}{This subsection establishes that
  identity precedes explanation, and explanation precedes interpretation,
  discharging REQ{DEP}.}

Together, interpretive non-commitment and
extension stability define the
functional boundaries of a neutral substrate.
These requirements establish that neutrality is
not a lack of content, but a categorical constraint:
to remain stable under pluralistic interpretation,
a foundational ontology must be
strictly pre-causal and pre-normative.
In the next section, we provide a
formal impossibility argument
showing why any violation of these categorical exclusions
renders a substrate inherently non-neutral.

\OBS{SCOPE-GUARD-DISCHARGE}{This subsection clarifies that neutrality
  is a role-relative constraint, not a universal requirement for all ontologies.}

\OBS{REQ-TO-IMP}{Section~\ref{sec:requirements} jointly defines neutrality
  such that any violation of pre-causality or pre-normativity
  entails failure of extension stability,
  enabling the impossibility argument in the next section.}
