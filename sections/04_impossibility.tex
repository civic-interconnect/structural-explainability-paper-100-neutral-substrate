% !TeX root = se100_the_ontological_neutrality_theorem.tex

\section{The Impossibility of Causal and Normative Substrates}
\label{sec:impossibility}

\REQ{IMPOSS-ROLE}{This section MUST formalize the neutrality requirements
  sufficiently to support a formal impossibility argument.}
\WHY{IMPOSS-ROLE}{The impossibility claim depends on
  precise functional requirements, not philosophical intuitions about neutrality.}

\REQ{IMPOSS-SCOPE}{The impossibility result MUST distinguish between
  structural substrate relations and
  framework-dependent causal or normative attributions.}
\WHY{IMPOSS-SCOPE}{The impossibility argument
  targets only ontological commitments
  that embed interpretive content,
  not those that merely support it.}

\OBS{IMPOSS-INDEP}{This section does not presuppose acceptance of the theorem;
  it supplies the definitions against which the theorem is evaluated.}


This section shows that the neutrality requirements
defined in Section~\ref{sec:requirements}
are incompatible with embedding causal or normative commitments
at the ontological substrate layer.
The argument is structural rather than empirical:
it does not depend on particular domains, datasets, or historical cases,
but on the logical relationship
between interpretive disagreement and ontological assertion.
All necessity and impossibility claims in this section
are relative to those requirements.


\subsection{Neutrality as Invariance Across LoAs}


Recall that a neutral ontological substrate,
by definition,
must be compatible with multiple admissible interpretive frameworks,
even when those frameworks disagree.
Compatibility here is understood in a strong sense:
the same substrate must remain logically consistent
when extended by divergent interpretive frameworks,
without revision or retraction of substrate-layer assertions.

A framework is \emph{admissible} if it is internally consistent and
represents a legitimate interpretive stance
within the domain of application, for example,
a recognized legal jurisdiction, an established causal modeling tradition,
or a coherent normative theory.
Admissibility is not universality:
admissible frameworks may contradict one another,
as when two jurisdictions reach opposite legal conclusions
or two causal models attribute responsibility differently.
The set of admissible frameworks $\mathbb{F}$ is thus
characterized by internal consistency, not mutual compatibility.

This neutrality requirement immediately constrains
the Level of Abstraction (LoA)
at which the substrate,
under the neutrality definition,
must operate.
For a substrate to be neutral,
its set of observables must be restricted to
those that are invariant across all
admissible interpretive frameworks.
Any ontological commitment that introduces
a higher-level observable, in particular one
requiring causal or normative attribution,
collapses the pluralism of the system into
a single, privileged framework,
thereby violating interpretive non-commitment.

\subsection{Normative Commitments and Interpretive Incompatibility}
\label{subsec:normative}

Normative statements, i.e., claims about
obligation, permission, prohibition, or violation,
are inherently framework-relative.
Their truth depends on legal regimes, institutional authorities,
temporal scope, and interpretive stance.
It is therefore a routine and expected feature
of accountability systems
that admissible frameworks disagree about normative conclusions.

Suppose an ontology asserts a normative conclusion at the substrate layer,
such as that a particular action was prohibited or
that an obligation applied to a given actor.
By doing so, the ontology privileges one normative framework over others.
Any framework that denies the asserted conclusion becomes
incompatible with the substrate,
as it cannot be layered atop it
without contradiction or revision.

This incompatibility is not a matter of
missing context or insufficient detail.
It arises from the ontological act of
asserting a contested normative conclusion as a fact.
As a result, no ontology that embeds normative commitments
at the foundational layer
can, under the neutrality requirements,
satisfy interpretive non-commitment.
Moreover, when alternative normative interpretations arise,
as they inevitably do,
the substrate must be revised to restore consistency,
thereby violating extension stability.

\OBS{NORMATIVE-DISCHARGE}{This subsection shows that normative commitments
  violate INC and EXT.}

\subsection{Causal Commitments and Model Dependence}
\label{subsec:causal}

Causal statements exhibit the same structural problem,
despite their different surface form.
Claims such as \textit{event A caused event B}
depend on background assumptions about causal mechanisms,
variable selection, counterfactual reasoning, and model scope.
Distinct causal frameworks may be equally admissible
while disagreeing about specific causal relationships.

If an ontology asserts causal relations at the substrate layer,
it necessarily commits to one causal model among many.
Frameworks that reject that model,
or that attribute causation differently,
cannot be layered onto the substrate without conflict.
As with normative commitments,
the ontology ceases to be interpretively neutral and,
under the extension stability requirement,
cannot remain stable under extension.

Importantly, this argument does not rely on
the presence of an active dispute over a specific event.
Rather, it identifies causation as a
category-type error for a neutral substrate.
Because causal attribution is inherently model-dependent,
its inclusion in the foundational layer
constitutes a structural commitment
to a specific counterfactual or mechanistic logic.
As soon as an ontology asserts $A \to B$ as a causal fact,
it excludes any framework that treats $A$ and $B$
as merely correlated or as having a common cause $Z$.
Therefore, under the neutrality requirements,
the substrate must be strictly pre-causal:
by treating a causal attribution, a function of an
interpretive model, as an ontological primitive,
the ontology conflates the object of observation
with its explanatory evaluation.

To be clear, \textit{pre-causal does not mean acausal}
(i.e., causation-denying).
The substrate does not deny that causation exists
or that causal reasoning is valuable.
Rather, it refrains from embedding causal conclusions
as foundational facts,
reserving causal attribution for interpretive layers
where framework assumptions are explicit
and disagreement is representable.
Causation is externalized, not eliminated.

\OBS{CAUSAL-DISCHARGE}{This subsection shows that
  causal commitments violate INC and EXT.}

\subsection{Reification Does Not Eliminate the Problem}
\label{subsec:reification}

One might attempt to preserve neutrality by
reifying causal or normative statements,
representing them as claims, reports, or assertions
rather than as direct ontological facts.
This move is necessary but not sufficient.

Reification allows the ontology to represent discourse about the world
without asserting conclusions about the world itself.
This strategy aligns with recent arguments for epistemic ontologies
that represent knowledge about the world
rather than the world directly~\cite{kassel2023epistemic}.
For example, an ontology may represent that
an agent asserted that event $A$ caused event $B$
without asserting the causal relation $A \rightarrow B$
as a substrate-layer fact.
In this way, reified claims are treated as entities
available for interpretation, comparison, or evaluation,
rather than as ontological commitments.

Reification preserves interpretive non-commitment
only if the ontology refrains from simultaneously
asserting the reified content as a substrate-layer truth.
If causal or normative relations are both asserted
as facts and reified as claims, neutrality is already lost.
If they are represented solely as claims,
then they no longer function as ontological commitments
but as objects of discourse layered atop the substrate.

Thus, reification does not provide a middle ground
in which causal or normative conclusions
can be embedded without consequence.
It either externalizes interpretation,
consistent with pre-causal and pre-normative design,
or it leaves the original incompatibility intact.

\OBS{REIFICATION-DISCHARGE}{This subsection shows that
  reification does not provide a middle ground for embedding
  causal or normative commitments.}

\subsection{The Ontological Neutrality Theorem}
\label{subsec:formal-proof}

The preceding observations yield the central result of this paper.

\begin{theorem}[Ontological Neutrality Theorem]
  \label{thm:neutrality}
  Let $\mathcal{S}$ be an ontology intended to function
  as a neutral substrate across diverse interpretive frameworks,
  and let $\mathbb{F}$ be the set of admissible interpretive frameworks.
  Then $\mathcal{S}$ satisfies the requirements of neutrality
  (interpretive non-commitment and extension stability)
  if and only if its foundational Level of Abstraction
  excludes causal and normative primitives.
\end{theorem}

\begin{proof}
  \textbf{(Only if.)}
  Suppose $\mathcal{S}$ contains a causal or deontic primitive $p$.
  By the definition of neutrality (Section~\ref{sec:requirements}),
  $\mathcal{S}$ must satisfy:
  for all $\mathcal{F} \in \mathbb{F}$,
  $\mathcal{S} \cup \mathcal{F} \not\vdash \bot$.

  Causal and normative primitives are framework-dependent:
  there exist admissible frameworks
  $\mathcal{F}_1, \mathcal{F}_2 \in \mathbb{F}$ such that
  $\mathcal{F}_1 \vdash p$ and
  $\mathcal{F}_2 \vdash \neg p$.

  Then $\mathcal{S} \cup \mathcal{F}_2 \vdash p \land \neg p \vdash \bot$.
  To restore consistency, either:
  \begin{enumerate}
    \item[(i)] $\mathcal{F}_i$ must be excluded from $\mathbb{F}$,
          violating interpretive non-commitment; or
    \item[(ii)] $\mathcal{S}$ must be revised to remove $p$,
          violating extension stability.
  \end{enumerate}
  In either case, $\mathcal{S}$ fails to satisfy neutrality.

  \medskip
  \textbf{(If.)}
  Suppose $\mathcal{S}$ excludes all causal and normative primitives,
  representing only entities and identity conditions.
  Under this setup,
  such observables are invariant across admissible frameworks:
  frameworks may disagree about why an entity exists or what it means,
  but not about the substrate-layer fact of its existence and identity.
  Thus, for all $\mathcal{F} \in \mathbb{F}$,
  $\mathcal{S} \cup \mathcal{F}$ introduces no contradiction,
  satisfying both interpretive non-commitment and extension stability.

  \medskip
  \noindent
  \textit{Scope note.}
  This direction assumes that entity existence and identity conditions
  are not themselves contested across admissible frameworks.
  In domains where identity is interpretively contested, for example,
  where frameworks disagree about whether two records denote the same entity,
  or whether a particular kind of entity exists at all,
  the substrate, under the neutrality definition,
  cannot be neutral until identity conditions are resolved.
  Such resolution is a precondition for neutrality, not a product of it.
  The theorem applies to domains where identity can be fixed
  prior to interpretive extension;
  it does not claim that all domains admit neutral substrates.
\end{proof}

This result is not contingent on particular modeling choices
or domain assumptions.
It follows directly from the functional role
the substrate is required to play under the neutrality definition.
Neutrality across disagreement and embedded interpretive commitments
are structurally incompatible.

\OBS{THEOREM-DISCHARGE}{This subsection formally proves
  the Ontological Neutrality Theorem, discharging the
  impossibility argument.}

In the next section,
we consider the implications of this result for ontology design,
clarifying what must be excluded from foundational layers
and how interpretation, evaluation, and explanation
may be externalized without loss of accountability.
