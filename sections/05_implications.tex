% !TeX root = se100_the_ontological_neutrality_theorem.tex

\section{Implications for Ontology Design}
\label{sec:implications}

\OBS{IMPL-ROLE}{This section derives design constraints
  that follow necessarily from the impossibility result,
  without proposing or endorsing any specific ontology.
  The purpose is constraint articulation, not construction or advocacy.}

\OBS{IMPL-INDEP}{This section may be reviewed independently
  as a consequence analysis of the impossibility theorem.}


The impossibility result established in Section~\ref{sec:impossibility}
does not prescribe a particular ontology.
Rather, it constrains the design space of any ontology
intended to function as a neutral substrate for accountability.
This section clarifies what such constraints require,
what they exclude, and
how interpretation and evaluation may be accommodated
without violating neutrality.

\OBS{IMPL-NONPRESCRIPTIVE}{This section does not introduce new requirements
  beyond those entailed by neutrality as defined earlier;
  all obligations stated here are logical consequences of the theorem,
  not additional design preferences.}

%-----------------------------------------------------
\subsection{What Must Be Excluded from the Substrate}
%-----------------------------------------------------

\OBS{EXCLUDE-CONSTRAINT}{From the theorem:
  the substrate must exclude causal, normative, and evaluative
  conclusions at the foundational level,
  as embedding such conclusions violates
  interpretive non-commitment and extension stability.}

The primary implication is negative but precise.
A neutral ontological substrate must exclude,
at the foundational level,
any assertions that entail propositions
whose truth values vary across admissible frameworks.
In particular, the substrate must not embed:
\begin{itemize}
  \item normative conclusions, such as obligations, permissions, prohibitions, violations, or compliance determinations;
  \item causal conclusions, such as assertions that one event caused, produced, or was responsible for another;
  \item evaluative judgments that privilege one legal, political, or analytic stance over others.
\end{itemize}

These exclusions do not deny the importance of such concepts for accountability.
Rather, they recognize that embedding them as ontological facts
collapses the distinction between representation and interpretation, undermining neutrality.

The exclusions apply specifically to asserted conclusions,
not to the entities or structures that such conclusions refer to.
Actors, events, institutional artifacts, jurisdictions,
and records may all be represented without committing to how they are interpreted or evaluated.
The substrate may therefore be rich in entities and identity structure
while remaining silent on contested interpretations.

\OBS{EXCLUDE-DISCHARGE}{This subsection discharges the negative constraints implied by neutrality.}

%-----------------------------------------------------
\subsection{What the Substrate Must Provide}
%-----------------------------------------------------

\OBS{PROVIDE}{From the theorem:
  the substrate must commit to entity existence,
  identity criteria, and persistence conditions,
  as accountability requires stable referents
  even in the presence of interpretive disagreement.}

Although the constraints are restrictive,
they do not leave the ontology impoverished.
On the contrary, a neutral substrate
must make strong, explicit commitments
in areas that are invariant across interpretive frameworks.
In terms of the Method of Levels of Abstraction (LoA)~\cite{floridi2008levels},
the substrate defines the set of \textit{invariant observables}:
data points that remain constant regardless
of the explanatory logic applied to them.

The claim that a neutral substrate relies on invariant observables
does not assert that all questions of
entity individuation, persistence, or boundary demarcation
are framework-independent.
Many such questions, e.g.,
the spatial or temporal granularity of an event,
the criteria for organizational continuity,
or the recognition of informal institutions,
are legitimately contested across admissible frameworks.
Neutrality instead requires that
the substrate fix only a minimal and domain-appropriate set of
identity and persistence conditions
sufficient to support accountability relations,
while deferring finer-grained individuation disputes
to interpretive layers.
These substrate-layer commitments are invariant
not because they resolve all ontological disagreement,
but because they remain stable under extension
by mutually incompatible causal, legal, or normative frameworks.

In particular, the substrate must provide:
\begin{itemize}
  \item stable reference to entities that participate in accountability relationships;
  \item clear identity criteria and disjointness conditions for those entities;
  \item persistence conditions that allow entities to be tracked across time, jurisdictions, and datasets.
\end{itemize}

These commitments are not optional.
Accountability depends on the ability
to refer unambiguously to who acted, what occurred,
which institutional instruments existed,
and within what jurisdictional scope.
Such commitments do not vary with interpretive stance
in the way causal or normative conclusions do,
and therefore do not threaten neutrality.

\OBS{PROVIDE-DISCHARGE}{This subsection identifies
  invariant commitments that do not threaten neutrality.}

%-----------------------------------------------------
\subsection{Externalizing Interpretation Without Loss}
%-----------------------------------------------------

\OBS{EXTERNALIZE}{From the theorem:
  interpretive, causal, and normative reasoning
  must be externalized to higher Levels of Organization,
  as externalization preserves neutrality
  while enabling pluralistic explanation and evaluation.}


A common concern is that excluding causal and normative commitments
from the substrate renders the ontology incapable of supporting
explanation, evaluation, or judgment.
The impossibility result shows that the opposite is true:
such functions must be externalized in order to be supported robustly.

Interpretation is accommodated by organizing the system
into distinct Levels of Organization (LoO)~\cite{floridi2011philosophy}.
The foundational LoO (the substrate)
records the what (entities and identity),
while higher-order LoOs record the how and why (interpretive claims).
By maintaining this structural separation,
we allow multiple, mutually inconsistent interpretive LoOs
to reside atop the same foundational LoO
without triggering logical collapse.
Disagreement, revision, and contestation are then represented
by the coexistence of multiple such records,
rather than by revision of substrate-layer facts.

Similarly, causal and normative reasoning may be carried out
within interpretive frameworks layered atop the substrate.
These frameworks may draw on the same underlying entities and events
while reaching incompatible conclusions.
The substrate remains unchanged,
serving as a stable point of reference
rather than a site of resolution.
This architecture does not diminish the importance of causal reasoning;
it locates causation where it belongs: in
interpretive layers where model commitments
are visible and contestable.

\OBS{EXTERNALIZE-DISCHARGE}{This subsection shows that exclusion from the substrate
  does not entail loss of expressive power.}

%-----------------------------------------------------
\subsection{Permitted Structural Relations and Excluded Commitments}
%-----------------------------------------------------

The exclusion of causal and normative primitives
does not entail the exclusion of all structured relations.
A neutral substrate permits relations that establish
structural, temporal, or institutional linkage
without asserting causal efficacy or evaluative force.
Relations such as containment, delegation, authorization,
temporal occurrence, participation, and applicability
are admissible insofar as they do not assert
that one entity caused, justified, or obligated another.
By contrast, relations that embed causal attribution
(e.g., X caused Y) or deontic judgment
(e.g., X violated Y, X was obligated to do Y)
are excluded from the substrate and must be represented,
if at all, via reification or within interpretive frameworks.
The neutrality boundary is thus drawn not
at the presence of structure,
but at the embedding of framework-dependent conclusions
as ontological facts.

%-----------------------------------------------------
\subsection{Reification as a Boundary Mechanism}
%-----------------------------------------------------

\OBS{REIFY}{From the theorem:
  reification must be used only to represent
  framework-dependent claims as entities,
  not to assert their truth at the substrate layer,
  as reification preserves neutrality only when endorsement is withheld.}


Reification plays a critical role in
enforcing the boundary between substrate and interpretation.
By treating claims, reports, and assertions
as entities in their own right,
the ontology can represent discourse about the world
without asserting conclusions about the world itself.

However, reification functions as a boundary mechanism
only when applied consistently.
If causal or normative relations are asserted as
substrate-layer facts and merely duplicated as reified claims,
neutrality is not preserved.
Reification supports neutrality only when
the substrate refrains entirely from endorsing the reified content.

Properly implemented, reification transforms
a framework-dependent predicate (e.g., $caused(A, B)$)
into an entity (e.g., $Assertion72$).
This shift in type allows the substrate
to maintain its identity-centric LoA
while providing the raw materials for
higher-level LoOs to perform their evaluative functions.

This observation reinforces the central conclusion:
neutrality is not achieved by representational cleverness alone,
but by disciplined exclusion of interpretive commitments from the foundational layer.

\OBS{REIFY-BOUNDARY}{Reification functions as a boundary mechanism,
  not a workaround for prohibited commitments.}

%-----------------------------------------------------
\subsection{Summary of Design Constraints}
%-----------------------------------------------------

The implications of the impossibility result can be summarized succinctly.
Any ontology intended to serve as a neutral substrate for accountability must:

\begin{itemize}
  \item restrict the foundational LoA to
        invariant observables (entities and identity conditions)
        while excluding interpretive predicates;

  \item prohibit the assertion of causal and normative
        conclusions as substrate-layer facts;

  \item preserve the LoO hierarchy to support interpretation,
        evaluation, and explanation exclusively through externalized,
        higher-level organizational layers;

  \item maintain monotonicity by remaining stable and unrevised
        under the addition of mutually incompatible interpretive frameworks.
\end{itemize}

\OBS{MINIMAL-DISCHARGE}{The substrate includes only invariant observables
  necessary for accountability; interpretive predicates are excluded,
  discharging REQ{MINIMAL}.}

These constraints do not define an ontology,
but they sharply delimit what any such ontology can be.%
\footnote{For illustration, consider a substrate that asserts
  \textit{Agent A's action caused Outcome B}.
  Under Framework $\mathcal{F}_1$ (a proximate-cause legal standard),
  this holds.
  Under Framework $\mathcal{F}_2$ (a but-for counterfactual model),
  it does not.
  The substrate must be revised to accommodate $\mathcal{F}_2$,
  violating extension stability.
  A pre-causal substrate would represent
  Agent A, the action, and Outcome B as entities,
  leaving causal attribution to the interpretive layer.}
In the following section, we conclude by
situating these constraints as boundary conditions
on any ontology intended to function as
a neutral substrate for accountability,
independent of domain or application.

\OBS{IMPL-CLOSURE}{This section completes the consequence analysis
  of the impossibility result and introduces no new assumptions.}
