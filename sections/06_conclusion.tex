% !TeX root = se100_the_ontological_neutrality_theorem.tex

\section{Conclusion}
\label{sec:conclusion}

This paper has argued that ontological neutrality,
when understood as interpretive non-commitment and
stability under incompatible extensions,
imposes strict and unavoidable constraints
on the design of ontological substrates.
By applying the Method of Levels of Abstraction (LoA),
we have demonstrated that any ontology intended to
function as a shared foundation for accountability
across disagreement must be pre-causal and pre-normative.

The argument is structural rather than empirical.
When causal or normative conclusions are asserted
as substrate-layer facts, neutrality is lost:
incompatible interpretive frameworks can no longer coexist
without revision or contradiction.
This incompatibility arises not from disagreement itself,
but from the Level of Organization (LoO)
at which contested commitments are embedded.
It can be resolved only by externalizing interpretation
to higher organizational layers that depend on,
but do not revise, the substrate.

The resulting design constraints are both restrictive and enabling.
They restrict what may be asserted at the foundational layer,
but they enable accountability by preserving stable reference to
entities and identity conditions across time and interpretive change.
Interpretation and explanation are not eliminated;
they are relocated to layers where disagreement
can be represented explicitly rather than suppressed
through premature ontological commitment.

At the same time, the impossibility result established here
is conditional rather than universal.
Neutral substrates are achievable only in domains
where entity existence and identity conditions
can be fixed prior to interpretive extension.
Where identity itself remains irreducibly contested,
neutrality is not merely difficult but unattainable:
the substrate cannot be stabilized without already
privileging one interpretive stance.
This limitation is not a defect of the framework,
but an explicit boundary condition on its applicability.

Taken together, these results clarify both the power
and the limits of ontological neutrality.
They establish a necessary condition that any ontology
intended to function as a neutral substrate must satisfy,
while also identifying the domain features
that make such neutrality possible.
By making these constraints explicit,
the paper delineates the space within which
ontological representations can remain neutral
in the presence of persistent interpretive disagreement
across institutional and analytic contexts.

This paper is deliberately meta-theoretical:
it establishes necessary constraints
without instantiating a particular ontology.
Concrete examples would introduce domain commitments
that compromise the generality of the result.
Demonstrating how these constraints apply to specific domains,
including the design of neutral substrates
for accountability and legal interoperability,
is an important direction for future work.
Exploring how existing upper ontologies
might be profiled or extended to satisfy these constraints
offers a promising path toward practical neutral substrates.

\section*{Acknowledgements}

Portions of this work were developed
using computer-assisted tools during manuscript preparation.
Generative language models were used to assist with editing, formatting,
and consistency checking during manuscript preparation.
All conceptual framing, formal development, results, interpretations, and conclusions
are the author's own.
All generated suggestions were critically reviewed and validated, and
the author takes full responsibility for the content of this work.
